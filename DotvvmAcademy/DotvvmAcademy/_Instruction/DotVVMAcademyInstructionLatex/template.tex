% Soubory musí být v kódování, které je nastaveno v příkazu \usepackage[...]{inputenc}

\documentclass[%
%  draft,    				  % Testovací překlad
  12pt,       				% Velikost základního písma je 12 bodů
  a4paper,    				% Formát papíru je A4
%  oneside,      			% Jednostranný tisk (výchozí)
%% Z následujicich voleb lze použít maximálně jednu:
%	dvipdfm  						% výstup bude zpracován programem 'dvipdfm' do PDF
%	dvips	  						% výstup bude zpracován programem 'dvips' do PS
%	pdftex							% překlad bude proveden programem 'pdftex' do PDF (výchozí)
%% Z následujících voleb lze použít jen jednu:
%english,            % originální jazyk je angličtina
%czech              % originální jazyk je čeština (výchozí)
slovak             % originální jazyk je slovenčina
]{report}				    	% Dokument třídy 'zpráva'

\usepackage[utf8]		%	Kódování zdrojových souborů je Windows-1250
	{inputenc}					% Balíček pro nastavení kódování zdrojových souborů

\usepackage{graphicx} % Balíček 'graphicx' pro vkládání obrázků
											% Nutné pro vložení log školy a fakulty


\usepackage{stackengine,xcolor}
\fboxrule=2pt
%\newcommand\cincludegraphics[2][]{%
  %\setbox0=\hbox{\includegraphics[#1]{#2}}%
  %\abovebaseline[-.5\ht0]{\includegraphics[#1]{#2}}}




\usepackage{forest}
\usepackage{dirtree} 

\usepackage{minted}

\listfiles



%\usepackage{caption}
%\floatsetup[listing]{style=Plaintop}
%\DeclareCaptionFont{white}{\scriptsize\color{white}}
%\DeclareCaptionFormat{listing}{\colorbox{gray}{\parbox{\linewidth-2\fboxsep}{#1#2#3}}}
%\captionsetup[listing]{format=listing,labelfont=white,textfont=white}
%\minted@define@extra{label}
\definecolor{bgColorCodeXsvikr}{RGB}{240, 240, 245} % Color for background of code
\definecolor{ForestGreenXsvikr}{RGB}{0, 150, 0} % Color for background of code
\usepackage[
	nohyperlinks				% Nebudou tvořeny hypertextové odkazy do seznamu zkratek
]{acronym}						% Balíček 'acronym' pro sazby zkratek a symbolů
											% Nutné pro použití prostředí 'seznamzkratek' balíčku 'thesis'

\usepackage[
	unicode,						% Záložky a informace budou v kódování unicode
	breaklinks=true,		% Hypertextové odkazy mohou obsahovat zalomení řádku
	hypertexnames=false % Názvy hypertextových odkazů budou tvořeny
											% nezávisle na názvech TeXu
]{hyperref}						% Balíček 'hyperref' pro sazbu hypertextových odkazů
											% Nutné pro použití příkazu 'nastavenipdf' balíčku 'thesis'

\usepackage{pdfpages} % Balíček umožňující vkládat stránky z PDF souborů
                      % Nutné při vkládání titulních listů a zadání přímo
                      % ve formátu PDF z informačního systému

\usepackage{enumitem} % Balíček pro nastavení mezerování v odrážkách
  \setlist{topsep=0pt,partopsep=0pt,noitemsep}

\usepackage{cmap} 		% Balíček cmap zajišťuje, že PDF vytvořené `pdflatexem' je
											% plně "prohledávatelné" a "kopírovatelné"

\usepackage{upgreek}	% Balíček pro sazbu stojatých řeckých písmem
											% např. stojaté pí: \uppi
											% např. stojaté mí: \upmu (použitelné třeba v mikrometrech)
											% pozor, grafická nekompatibilita s fonty typu Computer Modern!

%% Nastavení českého jazyka při sazbě v češtině.
% Pro sazbu češtiny je možné použít mezinárodní balíček 'babel', jenž
% použití doporučujeme pro nové instalace (MikTeX2.8,TeXLive2009), nebo
% národní balíček 'czech', který doporučujeme ve starších instalacích.
% Balíček 'babel' bude správně fungovat pouze ve spojení s programy
% 'latex', 'pdflatex', zatímco balíček 'czech' bude fungovat ve spojení
% s programy 'cslatex', 'pdfcslatex'.
% Varianta A:
\usepackage    				
  {babel}             % Balíček pro sazbu různojazyčných dokumentů; kompilovat (pdf)latexem!
  										% převezme si z parametrů třídy správný jazyk
\usepackage{lmodern}	% vektorové fonty Latin Modern, nástupce půvoních Knuthových Computern Modern fontů
\usepackage{textcomp} % Dodatečné symboly
\usepackage[T1]{fontenc}  % Kódování fontu - mj. kvůli správným vzorům pro dělení slov
% Varianta B:
%\usepackage{czech}   % Alternativní balíček pro sazbu v českém jazyce, kompilovat (pdf)cslatexem!




\usepackage[%
%% Z následujících voleb lze použít pouze jednu
%left,               % Rovnice a popisky plovoucich objektů budou %zarovnány vlevo
 center,             % Rovnice a popisky plovoucich objektů budou zarovnány na střed (vychozi)
%% Z následujících voleb lze použít pouze jednu
%semestral						%	sazba zprávy semestrálního projektu
bachelor						%	sazba bakalářské práce
%diploma						 % sazba diplomové práce
%treatise            % sazba pojednání o dizertační práci
%phd                 % sazba dizertační práce
]{thesis}             % Balíček pro sazbu studentských prací
                      % Musí být vložen až jako poslední, aby
                      % ostatní balíčky nepřepisovaly jeho příkazy

%%%%%%%%%%%%%%%%%%%%%%%%%%%%%%%%%%%%%%%%%%%%%%%%%%%%%%%%%%%%%%%%%
%%%%%%      Definice informací o dokumentu             %%%%%%%%%%
%%%%%%%%%%%%%%%%%%%%%%%%%%%%%%%%%%%%%%%%%%%%%%%%%%%%%%%%%%%%%%%%%


%%%%%%%%%%%%%%%%%%%%%%%%%%%%%%%%%%%%%%%%%%%%%%%%%%%%%%%%%%%%%%%%%%%%%%%

%%%%%%%%%%%%%%%%%%%%%%%%%%%%%%%%%%%%%%%%%%%%%%%%%%%%%%%%%%%%%%%%%%%%%%%
%%%%%%%%%%%       Začátek dokumentu               %%%%%%%%%%%%%%%%%%%%%
%%%%%%%%%%%%%%%%%%%%%%%%%%%%%%%%%%%%%%%%%%%%%%%%%%%%%%%%%%%%%%%%%%%%%%%
\begin{document}


\include{text/dirtree}

%% Vložení souboru 'text/vysledky' s popisem vysledků práce

\chapter{How to create (udpdate) lesson}
\label{sec:SRS}


\begin{enumerate}[label=\textbf{\arabic*}.]
%\setcounter{enumi}{5}
	\item Create new class \texttt{LessonZ.cs} in \textbf{Lessons} folder.\ref{FileTree}
\end{enumerate}
~\newline
\begin{minted}
[frame=topline,framerule=0.5mm,
label=Example LessonZ.cs,
framesep=2mm,
baselinestretch=1.2,
bgcolor=bgColorCodeXsvikr,
fontsize=\footnotesize,
linenos,breaklines
]{csharp}
public class LessonZ : LessonBase
{
	public LessonZ()
		{
			var lessonZXmlRelativePath = @"Lessons\LessonZ.xml";
			var lessonProvider = new LessonUserInterfaceProvider(lessonZRelativePath);
			Steps = lessonProvider.LessonSteps;
		}
}
\end{minted}
~\newline
~\newline
\begin{enumerate}[label=\textbf{\arabic*}.]
\setcounter{enumi}{1}
	\item In folder \textbf{Steps/Validation/Validators}\ref{FileTree} create folder \textbf{LessonZ} and create validator classes.
\end{enumerate}	
	~\newline
\begin{minted}
[frame=topline,framerule=0.5mm,
label=Example LessonZStepXValidator.cs,
framesep=2mm,
baselinestretch=1.2,
bgcolor=bgColorCodeXsvikr,
fontsize=\footnotesize,
linenos,breaklines
]{csharp}
namespace DotvvmAcademy.Steps.Validation.Validators.LessonZ
{
	[StepValidation(ValidatorKey = "LessonZStepXValidator")]
	public class LessonZStepXValidator: IDotHtmlCodeValidationObject
	{
		public void Validate(ResolvedTreeRoot resolvedTreeRoot)
		{
			ValidatorHelper.ValidateBasicControls(resolvedTreeRoot);
		}
	}
}
\end{minted}
~\newline
~\newline
\textsl{
By using reflection ValidatorProvider will be created new instance Validator Class (\texttt{LessonZStepXValidator.cs}) base on ValidatorKey from atribute. You can implement two interfaces for validate DotHtml (\texttt{IDotHtmlCodeValidationObject}) or C\# (\texttt{ICSharpCodeValidationObject}) syntax which is bound on type (Step Type="DothtmlCode") in XML.}


~\newline
\begin{enumerate}[label=\textbf{\arabic*}.]
\setcounter{enumi}{2}
	\item Now you can create xml doc \textsl{LessonY.xml} in \texttt{Lessons} folder.
\end{enumerate}

\begin{table}[h!]
\centering
\begin{tabular}{ccc}
\includegraphics[width=0.08\textwidth]{images/attention} & \begin{tabular}{@{}c@{}}\emph{Steps will be lined up according to \textbf{position} from XML doc!}\\ \emph {If you want to put new step between steps which already exist,} \\ \emph{you can simply create new step on the place between steps.} \end{tabular}   & \includegraphics[width=0.08\textwidth]{images/attention}\\ 
\end{tabular}
\end{table}  




\begin{minted}
[frame=topline,framerule=0.5mm,
label=Example LessonZ.xml,
framesep=2mm,
baselinestretch=1.2,
bgcolor=bgColorCodeXsvikr,
fontsize=\footnotesize,
linenos,breaklines
]{xml}
<?xml version="1.0" encoding="UTF-8"?>
<Lesson>
	<!-- namespace of all validators assigned to current lesson-->
	<ValidatorsNamespace>
	DotvvmAcademy.Steps.Validation.Validators.LessonZ
	</ValidatorsNamespace>
	<Steps>
		<!--Required attribute-->
		<Step Type="Info">
			<!--Required element-->
			<Title>XXXXXXX</Title>
			 <!--Required element-->
			<Description><![CDATA[XXXXXX]]></Description>
			<!--Optional element-->
			<ShadowBoxDescription><![CDATA[XXXXXX]]>
			</ShadowBoxDescription>
		</Step>
		<Step Type="CsharpCode">
			<Title>XXXXXXX</Title>
			<ValidationKey>XXXXXX</ValidationKey>
			<Description><![CDATA[XXXXXX]]></Description>
			<!--Optional element-->
			<ShadowBoxDescription><![CDATA[XXXXXX]]>
			</ShadowBoxDescription>
			<!--Required element-->
			<StartupCode><![CDATA[XXXXXX]]></StartupCode>
			<!--Required element-->
			<FinalCode><![CDATA[XXXXXX]]></FinalCode>
			<!--Optional element-->
			<CodeDependencies>
				<CodeDependency><![CDATA[XXXXXX]]>
				</CodeDependency>
				<CodeDependency><![CDATA[XXXXXX]]>
				</CodeDependency>
			</CodeDependencies>
			<!--Optional element-->
			<AllowedTypesConstructed>
			<AllowedType><![CDATA[XXXXXX]]></AllowedType>
			</AllowedTypesConstructed>
			<!--Optional element-->
			<AllowedMethodsCalled>
				<AllowedMethod><![CDATA[XXXXXX]]>
				</AllowedMethod>
			</AllowedMethodsCalled>
		</Step>
	</Steps>
</Lesson>
\end{minted}

\begin{table}[h!]
\centering
\begin{tabular}{ccc}
\includegraphics[width=0.08\textwidth]{images/warning} & \begin{tabular}{@{}c@{}}\emph{\texttt{ValidatorClass} must be in the same namespace like } \\ \emph{ \texttt{ValidatorsNamespace} in xml file}\end{tabular}   & \includegraphics[width=0.08\textwidth]{images/warning}\\ 
\end{tabular}
\end{table}

\begin{table}[h!]
\centering
\begin{tabular}{ccc}
\includegraphics[width=0.08\textwidth]{images/attention} & \begin{tabular}{@{}c@{}}\emph{If you create new validation message, please use resources file}\\ \emph{\textbf{Steps/Validation/Validators/ValidationErrorMessages.resx}} \\ \emph{Next point, in \texttt{DotvvmAcademy.Steps.Validation.Validator},} \\ \emph{contains class \texttt{ValidationExtensions}  with method \texttt{ExecuteSafe} } \\ \emph{which catch \texttt{RuntimeBinderException} exception} \end{tabular}   & \includegraphics[width=0.08\textwidth]{images/attention}\\ 
\end{tabular}
\end{table}  




\label{FileTree}
%\newline
\renewcommand*\DTstylecomment{\rmfamily\color{ForestGreenXsvikr}\textsc}
\renewcommand*\DTstyle{\ttfamily\textcolor{blue}}
\dirtree{%
.1 DotvvmAcademy.\DTcomment{Project}.
.2 ....
.2 Lessons.\DTcomment{folder with all Lessons}.
.3 Lesson1.cs.\DTcomment{lesson class}.
.3 Lesson1.xml.\DTcomment{lesson data}.
.3 ....
.3 LessonBase.cs.\DTcomment{lesson base abstract class}.
.2 ....
.2 Steps.
.3 StepBuilder.
.3 StepsBases.
.4 .....
.3 Validation.
.4 Interfaces.
.4 ValidatorProvision.
.4 Validators. 
.5 Lesson1.\DTcomment{folder with validator classes for lesson}.
.6 Lesson1Step3Validator.cs.\DTcomment{Validator class}.
.5 ....
.5 ValidationErrorMessages.resx.\DTcomment{Validation Error messages}.
.5 ValidationExtensions.cs.\DTcomment{Methods like ExecuteSafe}.
.5 ValidatorHelper.cs.
.4 ....
.3 CodeStepCsharp.cs.
.3 CodeStepDotHtml.cs.
.3 ChoicesStep.cs.
.3 ChoiceStepOption.cs.
.3 InfoStep.cs.
.2 ....
}
%~\newline



%% Konec dokumentu
\end{document}


\chapter{How to create (udpdate) lesson}
\label{sec:SRS}


\begin{enumerate}[label=\textbf{\arabic*}.]
%\setcounter{enumi}{5}
	\item Create new class \texttt{LessonZ.cs} in \textbf{Lessons} folder.\ref{FileTree}
\end{enumerate}
~\newline
\begin{minted}
[frame=topline,framerule=0.5mm,
label=Example LessonZ.cs,
framesep=2mm,
baselinestretch=1.2,
bgcolor=bgColorCodeXsvikr,
fontsize=\footnotesize,
linenos,breaklines
]{csharp}
public class LessonZ : LessonBase
{
	public LessonZ()
		{
			var lessonZXmlRelativePath = @"Lessons\LessonZ.xml";
			var lessonProvider = new LessonUserInterfaceProvider(lessonZRelativePath);
			Steps = lessonProvider.LessonSteps;
		}
}
\end{minted}
~\newline
~\newline
\begin{enumerate}[label=\textbf{\arabic*}.]
\setcounter{enumi}{1}
	\item In folder \textbf{Steps/Validation/Validators}\ref{FileTree} create folder \textbf{LessonZ} and create validator classes.
\end{enumerate}	
	~\newline
\begin{minted}
[frame=topline,framerule=0.5mm,
label=Example LessonZStepXValidator.cs,
framesep=2mm,
baselinestretch=1.2,
bgcolor=bgColorCodeXsvikr,
fontsize=\footnotesize,
linenos,breaklines
]{csharp}
namespace DotvvmAcademy.Steps.Validation.Validators.LessonZ
{
	[StepValidation(ValidatorKey = "LessonZStepXValidator")]
	public class LessonZStepXValidator: IDotHtmlCodeValidationObject
	{
		public void Validate(ResolvedTreeRoot resolvedTreeRoot)
		{
			ValidatorHelper.ValidateBasicControls(resolvedTreeRoot);
		}
	}
}
\end{minted}
~\newline
~\newline
\textsl{
By using reflection ValidatorProvider will be created new instance Validator Class (\texttt{LessonZStepXValidator.cs}) base on ValidatorKey from atribute. You can implement two interfaces for validate DotHtml (\texttt{IDotHtmlCodeValidationObject}) or C\# (\texttt{ICSharpCodeValidationObject}) syntax which is bound on type (Step Type="DothtmlCode") in XML.}


~\newline
\begin{enumerate}[label=\textbf{\arabic*}.]
\setcounter{enumi}{2}
	\item Now you can create xml doc \textsl{LessonY.xml} in \texttt{Lessons} folder.
\end{enumerate}

\begin{table}[h!]
\centering
\begin{tabular}{ccc}
\includegraphics[width=0.08\textwidth]{images/attention} & \begin{tabular}{@{}c@{}}\emph{Steps will be lined up according to \textbf{position} from XML doc!}\\ \emph {If you want to put new step between steps which already exist,} \\ \emph{you can simply create new step on the place between steps.} \end{tabular}   & \includegraphics[width=0.08\textwidth]{images/attention}\\ 
\end{tabular}
\end{table}  




\begin{minted}
[frame=topline,framerule=0.5mm,
label=Example LessonZ.xml,
framesep=2mm,
baselinestretch=1.2,
bgcolor=bgColorCodeXsvikr,
fontsize=\footnotesize,
linenos,breaklines
]{xml}
<?xml version="1.0" encoding="UTF-8"?>
<Lesson>
	<!-- namespace of all validators assigned to current lesson-->
	<ValidatorsNamespace>
	DotvvmAcademy.Steps.Validation.Validators.LessonZ
	</ValidatorsNamespace>
	<Steps>
		<!--Required attribute-->
		<Step Type="Info">
			<!--Required element-->
			<Title>XXXXXXX</Title>
			 <!--Required element-->
			<Description><![CDATA[XXXXXX]]></Description>
			<!--Optional element-->
			<ShadowBoxDescription><![CDATA[XXXXXX]]>
			</ShadowBoxDescription>
		</Step>
		<Step Type="CsharpCode">
			<Title>XXXXXXX</Title>
			<ValidationKey>XXXXXX</ValidationKey>
			<Description><![CDATA[XXXXXX]]></Description>
			<!--Optional element-->
			<ShadowBoxDescription><![CDATA[XXXXXX]]>
			</ShadowBoxDescription>
			<!--Required element-->
			<StartupCode><![CDATA[XXXXXX]]></StartupCode>
			<!--Required element-->
			<FinalCode><![CDATA[XXXXXX]]></FinalCode>
			<!--Optional element-->
			<CodeDependencies>
				<CodeDependency><![CDATA[XXXXXX]]>
				</CodeDependency>
				<CodeDependency><![CDATA[XXXXXX]]>
				</CodeDependency>
			</CodeDependencies>
			<!--Optional element-->
			<AllowedTypesConstructed>
			<AllowedType><![CDATA[XXXXXX]]></AllowedType>
			</AllowedTypesConstructed>
			<!--Optional element-->
			<AllowedMethodsCalled>
				<AllowedMethod><![CDATA[XXXXXX]]>
				</AllowedMethod>
			</AllowedMethodsCalled>
		</Step>
	</Steps>
</Lesson>
\end{minted}

\begin{table}[h!]
\centering
\begin{tabular}{ccc}
\includegraphics[width=0.08\textwidth]{images/warning} & \begin{tabular}{@{}c@{}}\emph{\texttt{ValidatorClass} must be in the same namespace like } \\ \emph{ \texttt{ValidatorsNamespace} in xml file}\end{tabular}   & \includegraphics[width=0.08\textwidth]{images/warning}\\ 
\end{tabular}
\end{table}

\begin{table}[h!]
\centering
\begin{tabular}{ccc}
\includegraphics[width=0.08\textwidth]{images/attention} & \begin{tabular}{@{}c@{}}\emph{If you create new validation message, please use resources file}\\ \emph{\textbf{Steps/Validation/Validators/ValidationErrorMessages.resx}} \\ \emph{Next point, in \texttt{DotvvmAcademy.Steps.Validation.Validator},} \\ \emph{contains class \texttt{ValidationExtensions}  with method \texttt{ExecuteSafe} } \\ \emph{which catch \texttt{RuntimeBinderException} exception} \end{tabular}   & \includegraphics[width=0.08\textwidth]{images/attention}\\ 
\end{tabular}
\end{table}  




\label{FileTree}
%\newline
\renewcommand*\DTstylecomment{\rmfamily\color{ForestGreenXsvikr}\textsc}
\renewcommand*\DTstyle{\ttfamily\textcolor{blue}}
\dirtree{%
.1 DotvvmAcademy.\DTcomment{Project}.
.2 ....
.2 Lessons.\DTcomment{folder with all Lessons}.
.3 Lesson1.cs.\DTcomment{lesson class}.
.3 Lesson1.xml.\DTcomment{lesson data}.
.3 ....
.3 LessonBase.cs.\DTcomment{lesson base abstract class}.
.2 ....
.2 Steps.
.3 StepBuilder.
.3 StepsBases.
.4 .....
.3 Validation.
.4 Interfaces.
.4 ValidatorProvision.
.4 Validators. 
.5 Lesson1.\DTcomment{folder with validator classes for lesson}.
.6 Lesson1Step3Validator.cs.\DTcomment{Validator class}.
.5 ....
.5 ValidationErrorMessages.resx.\DTcomment{Validation Error messages}.
.5 ValidationExtensions.cs.\DTcomment{Methods like ExecuteSafe}.
.5 ValidatorHelper.cs.
.4 ....
.3 CodeStepCsharp.cs.
.3 CodeStepDotHtml.cs.
.3 ChoicesStep.cs.
.3 ChoiceStepOption.cs.
.3 InfoStep.cs.
.2 ....
}
%~\newline

